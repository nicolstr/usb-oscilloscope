% Ersteller: 	Daniel Wirth
% Software:	TeX-Distribution --> MiKTeX (https://miktex.org/); Editor --> Texmaker (https://www.xm1math.net/texmaker/)
% Literatur:	.bib erstellt mit JabRef (https://www.jabref.org/); %Achtung: bilatex-mode verwenden (preferences) und dann .bib-file erstellen; Engine --> biber
% Der nachfolgende Praxisbericht wurde für das Praxisseminar im Studiengang BEI angefertigt.
%

\documentclass[a4paper, portrait, 12pt]{scrartcl} % Vorgabe TH Nbg
%%%%%%%%%%% Präambel %%%%%%%%%%%

% Standardpakete
%\usepackage[latin1]{inputenc}
\usepackage[T1]{fontenc}
\usepackage[ngerman]{babel}
\usepackage[left=3cm, right=2cm]{geometry} % Vorgabe TH Nbg
\usepackage{graphicx}
\graphicspath{ {./Abbildungen/} }
\usepackage{amsmath,amsfonts}
\usepackage{siunitx}
\usepackage{textcomp} % €
\usepackage[dvipsnames]{xcolor}
\usepackage{xspace} % Sorgt dafür, dass Leerzeichen hinter parameterlosen Makros nicht als Makroendezeichen interpretiert werden

% Formatierung etc.: Allgemein
\usepackage{setspace}
\onehalfspacing % Vorgabe TH Nbg
%\spacing{1.X}
\setlength{\parindent}{0pt} % Kein Einzug
\usepackage{afterpage}
\usepackage{lastpage}
\usepackage{fancyhdr} % Für beliebige Kopf- und Fußzeilen
\pagestyle{fancy}
\fancyhf{} % Löscht vorherige Einstellungen - Gesamt
%\fancyhead{} % Löscht voherige Einstellungen
%\fancyfoot{} % Dito
\renewcommand{\headrulewidth}{0 pt} % Keine obere Trennlinie
%\(this)pagestyle{empty} % Keine Seitenzahlen
%\setcounter{page}{n} % Bestimme Seite mit n bezeichnen, nachfolgende ist dann n+1
\rfoot{\thepage\ von \pageref{LastPage}} % Seite x von y; Vorgabe TH Nbg
\usepackage{enumitem}


% Formatierung etc.: Schrift
\usepackage{lmodern} % bessere Fonts
%\usepackage{ulem} % Unterstreichen
\usepackage{relsize} % Schriftgröße relativ festlegen


% Formatierung etc.: Tabellen/Arrays
%\usepackage{tabularx} % "Einfacher Tabllen gestalten"
%\usepackage{adjustbox}
\usepackage{caption} % Beschriftung von Bildern/Tabellen
\usepackage{subcaption} % Analog zu oben
%\setlength{\topsep}{0pt} % Kein xtra Abstand über Tabulator	
%\setlength{\partopsep}{0pt} % Kein xtra Abstand unter Tabulator	


% Hyperlinks und URLs etc.
\usepackage{hyperref}
\hypersetup{colorlinks=true, linkcolor=black, urlcolor=MidnightBlue, citecolor=black}
\urlstyle{same}


% Graphiken, Plots, PDFs etc.
%!TEX root = name.tex % Andere .tex-Datei einbinden
\usepackage{pdfpages}
\usepackage{float}
\usepackage{wrapfig}
%\usepackage{pgfplots}
%\pgfplotsset{width=10cm,compat=1.9} % Zweiten Parameter nicht verändern!


% Rechtschreibung bzw. Trennung
%\hyphenchar\font=\string"7F % Nötig für korrekte Silbentrennung!
%\hyphenation{Kom-pli-ziert-es-Wort}
%\hyphenpenalty = XXX
%\tolerance = XXX


% Mathe
%\usepackage{aligned-overset}
%\usepackage{mathabx} %\widehat{•} & \widecheck{}


% "Speziellere" Befehle
%\newcommand{\forceindent}{\leavevmode{\parindent=1.5em\indent}}


% Bibliographie
\usepackage[style=alphabetic, backend=biber]{biblatex}  
\usepackage{csquotes} % Für Anführungszeichen bei Zitaten
\addbibresource{Bibliographie_Projekt_USB_Oszi.bib}
%\usepackage[nohyperlinks, printonlyused]{acronym} % Abkürzungsverzeichnis


\title{Projekt USB-Oszilloskop}
\author{Samuel Oeser, Nicole Sturm, Daniel Wirth}
\date{\today}


%%%%%%%%%%% Main %%%%%%%%%%%
\begin{document}

%%%%%%%%%%% Deckblatt %%%%%%%%%%%
\maketitle
%\thispagestyle{empty}
\pagebreak

%%%%%%%%%%% TOC %%%%%%%%%%%
%\thispagestyle{empty}
\tableofcontents
\pagebreak

%%%%%%%%%%% Abstract / Zusammenfassung %%%%%%%%%%%
\section{Abstract / Zusammenfassung}

\pagebreak

%%%%%%%%%%% Einleitung %%%%%%%%%%%
\section{Einleitung}

\pagebreak

%%%%%%%%%%% Fachliche Grundlagen %%%%%%%%%%%
\section{Fachliche Grundlagen}

\subsection{Allgemeiner Aufbau eines DSOs}
Mühl: \cite[Abb. 14.1]{Muehl2020}
\subsection{Leitungsimpedanzanpassung}

\subsection{ADC-Topologien}

\subsection{AAF-Entwurf (Nyquisttheorem)}

\subsection{Frequenzkompensierter Spannungsteiler}
Schrüfer: \cite[S. 114ff]{Schruefer2022}
\subsection{Zustandsautomat (Finite State-Machine - FSM)}

\subsection{Direct Memory Access - DMA}

\subsection{Digitale Filterung (Preprocessing)}

\pagebreak

%%%%%%%%%%% Projektkonzeption %%%%%%%%%%%
\section{Projektkonzeption}

\subsection{Vorgehensweise}

\subsection{Anforderungen}

\subsection{Konzept}


\pagebreak

%%%%%%%%%%% Realisierung %%%%%%%%%%%
\section{Realisierung}

\subsection{Hardware (HW)}
\subsubsection{Entwurf}
\subsubsection{Implementierung}
\subsubsection{HW-Test}

\subsection{Schnittstelle Hardware - Firmware}


\subsection{Firmware (FW)}
\subsubsection{Entwurf}
\subsubsection{Implementierung}
\subsubsection{FW-Test}

\subsection{Schnittstelle Firmware - Software}

\subsection{Software (SW)}

\subsection{Zusammenführung}
\subsubsection{Entwurf}
\subsubsection{Implementierung}
\subsubsection{SW-Test}

\pagebreak

%%%%%%%%%%% Schluss %%%%%%%%%

\pagebreak

%%%%%%%%%%% Ergebnisse %%%%%%%%%%%
\section{Ergebnisse}

\pagebreak

%%%%%%%%%%% Fazit und Ausblick %%%%%%%%%%%
\section{Fazit und Ausblick}

\pagebreak

%%%%%%%%%%% Literaturverzeichnis %%%%%%%%%%%
\section{Literaturverzeichnis}
\printbibliography
\pagebreak

%%%%%%%%%%% Abbildungsverzeichnis %%%%%%%%%%%
\section{Abbildungsverzeichnis}
\listoffigures
\pagebreak

%%%%%%%%%%% Abkürzungsverzeichnis %%%%%%%%%%%
%\section{Abkürungsverzeichnis}

%\begin{acronym}
%\acro{Abk.}{Abkürzung} % Im Text mit \ac{Abk.} verwenden
%\end{acronym}

%\pagebreak

%%%%%%%%%%% Anhang %%%%%%%%%%%
\appendix
\section{Anhang}

  
\end{document}
